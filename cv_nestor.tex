%!TEX TS-program = xelatex
%!TEX encoding = UTF-8 Unicode
% Awesome CV LaTeX Template for CV/Resume
%
% This template has been downloaded from:
% https://github.com/posquit0/Awesome-CV
%
% Author:
% Claud D. Park <posquit0.bj@gmail.com>
% http://www.posquit0.com 
%
% Template license: 
% CC BY-SA 4.0 (https://creativecommons.org/licenses/by-sa/4.0/)
%


%-------------------------------------------------------------------------------
% CONFIGURATIONS
%-------------------------------------------------------------------------------
% A4 paper size by default, use 'letterpaper' for US letter
\documentclass[11pt, a4paper]{awesome-cv}

% Configure page margins with geometry
\geometry{left=1.4cm, top=.8cm, right=1.4cm, bottom=1.8cm, footskip=.5cm}

% Specify the location of the included fonts
\fontdir[fonts/]

% Color for highlights
% Awesome Colors: awesome-emerald, awesome-skyblue, awesome-red, awesome-pink, awesome-orange
%                 awesome-nephritis, awesome-concrete, awesome-darknight
\colorlet{awesome}{awesome-red} 
% Uncomment if you would like to specify your own color
% \definecolor{awesome}{HTML}{CA63A8}

% Colors for text
% Uncomment if you would like to specify your own color
% \definecolor{darktext}{HTML}{414141}
% \definecolor{text}{HTML}{333333}
% \definecolor{graytext}{HTML}{5D5D5D}
% \definecolor{lighttext}{HTML}{999999}

% Set false if you don't want to highlight section with awesome color
\setbool{acvSectionColorHighlight}{true}

% If you would like to change the social information separator from a pipe (|) to something else
\renewcommand{\acvHeaderSocialSep}{\quad\textbar\quad}


%-------------------------------------------------------------------------------
%	PERSONAL INFORMATION
%	Comment any of the lines below if they are not required
%-------------------------------------------------------------------------------
% Available options: circle|rectangle,edge/noedge,left/right
% \photo{profile.png}
\name{Nestor D.}{Pereira Neto}
\position{Electronic engineer{\enskip\cdotp\enskip}36}
\address{Street Eduardo Campos n$^{\circ}$10, Boca do Rio, Salvador - BA, Brazil, 41705-230}

\mobile{(71) 9 9698-6755}
\email{nestor-dp@hotmail.com}
%\homepage{www.posquit0.com}
\github{NestorDP}
\linkedin{nestorpneto}
%\gplus{+NestorNeto85}
\lattes{lattes.cnpq.br/8490310764316084}

% \gitlab{gitlab-id}
% \stackoverflow{SO-id}{SO-name}
% \twitter{@twit}
% \skype{skype-id}
% \reddit{reddit-id}
% \extrainfo{extra informations}

%\quote{``Be the change that you want to see in the world."}

\quote{``I have a degree in Electronic Engineering (2013) and a specialization in Biomedical Engineering (2019). I have experience in the maintenance of diagnostic imaging hospital equipment. In 2011, I started to work as a teacher in technical and higher education courses. I am currently finishing my Masters in Electrical Engineering at the Federal University of Bahia in the area of computing and robotics. Since 2019 I have had the pleasure of working in research and development in the field of robotics. Main areas of interest: C/C++ Programming; Embedded systems; Real-time systems and Mobile robotics localization and navigation.''}
%-------------------------------------------------------------------------------
\begin{document}

% Print the header with above personal informations
% Give optional argument to change alignment(C: center, L: left, R: right)
\makecvheader

% Print the footer with 3 arguments(<left>, <center>, <right>)
% Leave any of these blank if they are not needed
\makecvfooter
  {\today}
  {Nestor D. Pereira Neto~~~·~~~Curriculum Vitae}
  {\thepage}


%-------------------------------------------------------------------------------
%	CV/RESUME CONTENT
%	Each section is imported separately, open each file in turn to modify content
%-------------------------------------------------------------------------------
%-------------------------------------------------------------------------------
%	SECTION TITLE
%-------------------------------------------------------------------------------
\cvsection{Formação}


%-------------------------------------------------------------------------------
%	CONTENT
%-------------------------------------------------------------------------------
\begin{cventries}

%---------------------------------------------------------
  \cventry
    {Mestrado em Engenharia Elétrica} % 
    {Escola Politécnica - Universidade Federal da Bahia} % Institution
    {Salvador, BA} % Location
    {Abr. 2018 - Exp. Abr. 2020} % Date(s)
    {
      \begin{cvitems} % Description(s) bullet points
        \item {Linha de pesquisa: Computação e Robótica.}
        \item {Coprocessador de vídeo em FPGA para integração com \textit{Robot Operating System - ROS}.}
      \end{cvitems}
    }

%---------------------------------------------------------

%---------------------------------------------------------
  \cventry
    {Especialização em Engenharia Biomédica com Ênfase em Engenharia Clínica} % 
    {Universidade Estácio de Sá} % Institution
    {Salvador, BA} % Location
    {Out. 2017 - Exp. Fev. 2019} % Date(s)
    {
      \begin{cvitems} % Description(s) bullet points
        \item {Rede Neural Convolucional para Detecção de Complexos QRS em Sinais de Eletrocardiograma}
      \end{cvitems}
    }

%---------------------------------------------------------

%---------------------------------------------------------
  \cventry
    {Bacharelado em engenharia elétrica - habilitação eletrônica} % Degree
    {Faculdade de Ciência e Tecnologia - ÁREA1} % Institution
    {Salvador, BA} % Location
    {Fev. 2008 - Jul. 2013} % Date(s)
    {
      \begin{cvitems} % Description(s) bullet points
        \item {Protótipo de comando para enquadrar equipamentos de raio X às normas exigidas pela ANVISA.}
        \item {Bolsista de iniciação científica}
        \item {Bolsista do programa Universidade para Todos - Prouni.}
      \end{cvitems}
    }

%---------------------------------------------------------

\end{cventries}

%-------------------------------------------------------------------------------
%	SECTION TITLE
%-------------------------------------------------------------------------------
\cvsection{Experiências Profissionais}


%-------------------------------------------------------------------------------
%	CONTENT
%-------------------------------------------------------------------------------
\begin{cventries}

%---------------------------------------------------------
  \cventry
    {Consultor II - Robótica} % Job title
    {SENAI - CIMATEC} % Organization
    {Salvador, BA} % Location
    {Abr. 2019 - Atual} % Date(s)
    {
      \begin{cvitems} % Description(s) of tasks/responsibilities
        \item {Dimensionamento dos dispositivos sistemas de potência (fontes de alimentação, conversores AC/DC e DC/DC e baterias).}
        \item {Desenvolvimento de hardware e firmware de Sistemas Embarcados.}
        \item {Programação framework de robótica ROS, python e C/C++.}
      \end{cvitems}
    }

%---------------------------------------------------------
  \cventry
    {Professor horista - 20h} % Job title
    {} % Organization
    {} % Location
    {Nov. 2018 - Abr. 2019} % Date(s)
    {
      \begin{cvitems} % Description(s) of tasks/responsibilities
        \item {Ministrar aulas para o curso: Mecatrônica Industrial (CQPG).}
        \item {Disciplina: Acionamento de Dispositivos e Atuadores, Processamento de sinais, Eletrônica analógica e digital.}
      \end{cvitems}
    }

%---------------------------------------------------------
%  \cventry
%    {Professor horista} % Job title
%    {SENAI - CETIND} % Organization
%    {Lauro de Freitas, BA} % Location
%    {Set. 2017 - Dez. 2017} % Date(s)
%    {
%      \begin{cvitems} % Description(s) of tasks/responsibilities
%        \item {Ministrar aulas para o cursos: Técnico em Mecânica (CHP) e Técnico em Eletroeletrônica (CHP).}
%        \item {Disciplinas: Eletrônica Básica - 32h e Eletrotécnica - 88h.}
%      \end{cvitems}
%    }
    
%---------------------------------------------------------
  \cventry
    {Técnico de laboratório} % Job title
    {Faculdade ÁREA1|Wyden} % Organization
    {Salvador, BA} % Location
    {Jul. 2013 - Ago. 2017} % Date(s)
    {
      \begin{cvitems} % Description(s) of tasks/responsibilities
        \item {Elaborar e ministras aulas práticas para os cursos: Engenharia da Computação, Engenharia Elétrica e Engenharia de Automação.}
        \item {Testar novas ferramentas utilizadss nos laboratórios e ministrar treinamento para os professores.}
        \item { Ministrar cursos de extensão e aulas de carga horária complementar do programa de experiências - PEX.}
      \end{cvitems}
    }

%---------------------------------------------------------
  \cventry
    {Engenheiro Eletricista} % Job title
    {Alfamed Eletromedicina} % Organization
    {Salvador, BA} % Location
    {Ago. 2013 - Fev. 2015} % Date(s)
    {
      \begin{cvitems} % Description(s) of tasks/responsibilities
        \item {Responsável técnico pela empresa junto ao CREA-BA.}
        \item {Manutenção eletroeletrônica de equipamentos médico-hospitalares.}
        \item {Desenvolvimento de novos produtos e soluções.}
      \end{cvitems}
    }
    
%---------------------------------------------------------
%  \cventry
%    {Professor Temporário} % Job title
%    {CPC Treinamentos} % Organization
%    {Camaçari, BA} % Location
%    {Jan. 2013 - jun. 2013} % Date(s)
%    {
%      \begin{cvitems} % Description(s) of tasks/responsibilities
%        \item {Ministrar aulas das disciplinas: Eletricidade I, Eletrônica Analógica  e Eletrônica Digital.}
%      \end{cvitems}
%    }

%---------------------------------------------------------
  \cventry
    {Professor - 40h} % Job title
    {Centro Territorial de Educação Profissional da Região Metropolitana CETEP-RM} % Organization
    {Camaçari, BA} % Location
    {jun. 2011 - jun. 2013} % Date(s)
    {
      \begin{cvitems} % Description(s) of tasks/responsibilities
        \item {Ministrar aula das disciplinas: Eletrônica analógica e digital; Microcontroladores; Circuitos elétricos.}
      \end{cvitems}
    }
    
%---------------------------------------------------------
%  \cventry
%    {Analista Técnico} % Job title
%    {TELTRONIC-Brasil} % Organization
%    {Salvador, BA} % Location
%    {Fev. 2011 - Jun. 2011} % Date(s)
%    {
%      \begin{cvitems} % Description(s) of tasks/responsibilities
%        \item {Manutenção corretiva em terminais digitais (móveis e fixos) de comunicação profissional.}
%      \end{cvitems}
%    }
    
%---------------------------------------------------------
  \cventry
    {Técnico de manutenção} % Job title
    {Micro Comércio e Serviço Ltda.} % Organization
    {Salvador, BA} % Location
    {Jun. 2009 - Set. 2010} % Date(s)
    {
      \begin{cvitems} % Description(s) of tasks/responsibilities
        \item {Manutenção preventiva e corretiva em equipamentos hospitalares de bioimagem.}
      \end{cvitems}
    }

%---------------------------------------------------------
  \cventry
    {Estagiário - manutenção} % Job title
    {Micro Comércio e Serviço Ltda.} % Organization
    {Salvador, BA} % Location
    {Mar. 2008 - Jun. 2009} % Date(s)
    {
      \begin{cvitems} % Description(s) of tasks/responsibilities
        \item {Manutenção preventiva e corretiva em equipamentos hospitalares de bioimagem.}
      \end{cvitems}
    }

%---------------------------------------------------------
%  \cventry
%    {Freelance Penetration Tester} % Job title
%    {SAMSUNG Electronics} % Organization
%    {S.Korea} % Location
%    {Sep. 2013, Mar. 2011 - Oct. 2011} % Date(s)
%    {
%      \begin{cvitems} % Description(s) of tasks/responsibilities
%        \item {Conducted penetration testing on SAMSUNG KNOX, which is solution for enterprise mobile security.}
%        \item {Conducted penetration testing on SAMSUNG Smart TV.}
%      \end{cvitems}
      %\begin{cvsubentries}
      %  \cvsubentry{}{KNOX(Solution for Enterprise Mobile Security) Penetration Testing}{Sep. 2013}{}
      %  \cvsubentry{}{Smart TV Penetration Testing}{Mar. 2011 - Oct. 2011}{}
      %\end{cvsubentries}
%    }

%---------------------------------------------------------
\end{cventries}

%-------------------------------------------------------------------------------
%	SECTION TITLE
%-------------------------------------------------------------------------------
\cvsection{Idiomas}


%-------------------------------------------------------------------------------
%	CONTENT
%-------------------------------------------------------------------------------
\begin{cvskills}

%---------------------------------------------------------
  \cvskill
    {Português} % Category
    {Native.} % Skills

%---------------------------------------------------------
  \cvskill
    {Inglês} % Category
    {A2 level - CERF.} % Skills
    
%---------------------------------------------------------
%  \cvskill
%    {Espanhol} % Category
%    {Nível intermediário.} % Skills


%---------------------------------------------------------
\end{cvskills}


%---------------------------------------------------------
%  \cvskill
%    {Programming} % Category
%    {Python, Node.JS, C/C++, Scala, JAVA, OCaml, LaTeX} % Skills

%---------------------------------------------------------
%  \cvskill
%    {Web} % Category
%    {Django, Express, Redux, React, HTML5, LESS, SASS} % Skills

%---------------------------------------------------------
%  \cvskill
%    {Languages} % Category
%    {Korean, English, Japanese, Chinese} % Skills
%-------------------------------------------------------------------------------
%	SECTION TITLE
%-------------------------------------------------------------------------------
\cvsection{Courses}


%-------------------------------------------------------------------------------
%	CONTENT
%-------------------------------------------------------------------------------
\begin{cvhonors}
%---------------------------------------------------------
\cvhonor
  {NR10 - Sistemas Elétricos de Potência - SEP (40)} % Position
  {IN1000} % Committee
  {Salvador, BA} % Location
  {2022} % Date(s)

%---------------------------------------------------------
\cvhonor
  {Segurança em Eletricidade - NR10(40)} % Position
  {SENAI} % Committee
  {Salvador, BA} % Location
  {2022} % Date(s)

%---------------------------------------------------------
\cvhonor
  {FPGA INTEL Training(20h)} % Position
  {Macnica DHW - Official Training Center FPGA INTEL} % Committee
  {Florianópolis, SC} % Location
  {2021} % Date(s)

%---------------------------------------------------------
\cvhonor
  {NucLi - English. listening comprehension (32h)} % Position
  {Federal University of Bahia - UFBA} % Committee
  {Salvador, BA} % Location
  {2018} % Date(s)
    
%---------------------------------------------------------
\cvhonor
  {Programa Idiomas sem Fronteiras: My English Online - Level 4 (120h) } % Position
  {Ministry of Education - MEC.} % Committee
  {Brasil} % Location
  {2018} % Date(s)

%---------------------------------------------------------
\cvhonor
  {Study of Device Control via parallel, serial and USB ports (60h)} % Curso
  {Faculdade ÁREA1.} % Instituição
  {Salvador, BA} % Location
  {2010} % Date(s)

%---------------------------------------------------------
\cvhonor
  {Applied Analog Electronics (30h)} % Curso
  {Faculdade ÁREA1.} % Instituição
  {Salvador, BA} % Location
  {2010} % Date(s)

%---------------------------------------------------------
\end{cvhonors}

%-------------------------------------------------------------------------------
%	SECTION TITLE
%-------------------------------------------------------------------------------
\cvsection{Projetos e Pesquisa}


%-------------------------------------------------------------------------------
%	CONTENT
%-------------------------------------------------------------------------------
\begin{cventries}

%---------------------------------------------------------
  \cventry
    {Projeto de mestrado} % Affiliation/role
    {Universidade Federal da Bahia - UFBA} % Organization/group
    {Salvador, BA} % Location
    {Dez. 2018 - atualmente} % Date(s)
    {
      \begin{cvitems} % Description(s) of experience/contributions/knowledge
        \item {Desenvolvimento de um coprocessador de vídeo para integração com framework de robótica - ROS.}
        \item {Projeto de hardware em verilog para implementação em FPGA (Cyclone IV - Intel).}
        \item {Programação em liguagem C do soft processador NiosII.}
        \item {Aplicação de sistema de tempo real através do FreeRTOS.}
        \item {Desenvolvimento de nodes ROS em linguagem C/C++ e Python.}
      \end{cvitems}
    }

%---------------------------------------------------------
  \cventry
    {Bolsista de iniciação científica} % Affiliation/role
    {Faculdade ÁREA1} % Organization/group
    {Salvador, BA} % Location
    {Jul. 2011 - Jul. 2013} % Date(s)
    {
      \begin{cvitems} % Description(s) of experience/contributions/knowledge
        \item {Desenvolvimento de um comando microcontrolado para enquadrar equipamentos de raios X antigos às normas técnicas atuais.}
        \item {Programação de microcontroladores da família 8051 em linguagem assembly.}
        \item {Pesquisa das normas técnicas estabelecidas pela ANVISA em relação a equipamentos de raios X.}
        \item {Apresentação de relatórios mensais com as atividades desenvolvidas no período.}
        \item {Puplicação dos resultados em artigo na revista \underline{\href{https://cientefico.emnuvens.com.br/cientefico/article/view/58}{\emph{Cientefico ISSN 1677-1591, jul/dez 2013}}}.}
      \end{cvitems}
    }
    

%---------------------------------------------------------
\end{cventries}


%-------------------------------------------------------------------------------
%	SECTION TITLE
%-------------------------------------------------------------------------------
\cvsection{Competências e Habilidades}


%-------------------------------------------------------------------------------
%	CONTENT
%-------------------------------------------------------------------------------
\begin{cvskills}

%---------------------------------------------------------
  \cvskill
    {Básico} % Category
    {Verilog HDL, POO, Sockets, GDB, Git.} % Skills

%---------------------------------------------------------
  \cvskill
    {Intermediário} % Category
    {Python, ARM Cortex M, Linux, Windows, PIC, AutoCAD, ROS/Gazebo, GCC, Make, Nios II, FreeRTOS.} % Skills

%---------------------------------------------------------
  \cvskill
    {Avançado} % Category
    {Layout de circuito impresso - KiCad, Matlab, C/C++, AVR, 8051, Manutenção eletrônica, LaTeX.} % Skills

%---------------------------------------------------------
\end{cvskills}


%---------------------------------------------------------
%  \cvskill
%    {Programming} % Category
%    {Python, Node.JS, C/C++, Scala, JAVA, OCaml, LaTeX} % Skills

%---------------------------------------------------------
%  \cvskill
%    {Web} % Category
%    {Django, Express, Redux, React, HTML5, LESS, SASS} % Skills

%---------------------------------------------------------
%  \cvskill
%    {Languages} % Category
%    {Korean, English, Japanese, Chinese} % Skills
%%-------------------------------------------------------------------------------
%	SECTION TITLE
%-------------------------------------------------------------------------------
\cvsection{Honors \& Awards}


%-------------------------------------------------------------------------------
%	SUBSECTION TITLE
%-------------------------------------------------------------------------------
% \cvsubsection{International}


%-------------------------------------------------------------------------------
%	CONTENT
%-------------------------------------------------------------------------------
\begin{cvhonors}

%---------------------------------------------------------
  \cvhonor
    {PETROBRAS} % Award
    {Prêmio Inventor 2022 - Award for filing a patent for an underwater 3D scanner} % Event
    {Rio de Janeiro, Brasil} % Location
    {2022} % Date(s)


%---------------------------------------------------------
\end{cvhonors}


%-------------------------------------------------------------------------------
%	SUBSECTION TITLE
%-------------------------------------------------------------------------------
% \cvsubsection{Domestic}


%-------------------------------------------------------------------------------
%	CONTENT
%-------------------------------------------------------------------------------
% \begin{cvhonors}

%---------------------------------------------------------
%   \cvhonor
%     {3rd Place} % Award
%     {WITHCON Hacking Competition Final} % Event
%     {Seoul, S.Korea} % Location
%     {2015} % Date(s)


% \end{cvhonors}

%-------------------------------------------------------------------------------
%	SECTION TITLE
%-------------------------------------------------------------------------------
\cvsection{Presentation}


%-------------------------------------------------------------------------------
%	CONTENT
%-------------------------------------------------------------------------------
\begin{cventries}

%---------------------------------------------------------
  \cventry
    {Oral Presentation} % Role
    {28th IBERCHIP Workshop} % Event
    {Santiago, Chile} % Location
    {1-4 Mar. 2022} % Date(s)
    {
      \begin{cvitems} % Description(s)
        \item {Comunicação entre Robot Operating System - ROS e SoC com FPGA integrado}
      \end{cvitems}
    }

%---------------------------------------------------------
  % \cventry
  %   {Presenter for <Metasploit 101>} % Role
  %   {6th Hacking Camp - S.Korea} % Event
  %   {S.Korea} % Location
  %   {Sep. 2012} % Date(s)
  %   {
  %     \begin{cvitems} % Description(s)
  %       \item {Introduced basic procedure for penetration testing and how to use Metasploit}
  %     \end{cvitems}
  %   }

%---------------------------------------------------------
\end{cventries}

%%-------------------------------------------------------------------------------
%	SECTION TITLE
%-------------------------------------------------------------------------------
\cvsection{Writing}


%-------------------------------------------------------------------------------
%	CONTENT
%-------------------------------------------------------------------------------
\begin{cventries}

%---------------------------------------------------------
  \cventry
    {Founder \& Writer} % Role
    {A Guide for Developers in Start-up} % Title
    {Facebook Page} % Location
    {Jan. 2015 - PRESENT} % Date(s)
    {
      \begin{cvitems} % Description(s)
        \item {Drafted daily news for developers in Korea about IT technologies, issues about start-up.}
      \end{cvitems}
    }

%---------------------------------------------------------
  \cventry
    {Undergraduate Student Reporter} % Role
    {AhnLab} % Title
    {S.Korea} % Location
    {Oct. 2012 - Jul. 2013} % Date(s)
    {
      \begin{cvitems} % Description(s)
        \item {Drafted reports about IT trends and Security issues on AhnLab Company magazine.}
      \end{cvitems}
    }

%---------------------------------------------------------
\end{cventries}




%-------------------------------------------------------------------------------
\end{document}

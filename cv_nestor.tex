%!TEX TS-program = xelatex
%!TEX encoding = UTF-8 Unicode
% Awesome CV LaTeX Template for CV/Resume
%
% This template has been downloaded from:
% https://github.com/posquit0/Awesome-CV
%
% Author:
% Claud D. Park <posquit0.bj@gmail.com>
% http://www.posquit0.com 
%
% Template license: 
% CC BY-SA 4.0 (https://creativecommons.org/licenses/by-sa/4.0/)
%


%-------------------------------------------------------------------------------
% CONFIGURATIONS
%-------------------------------------------------------------------------------
% A4 paper size by default, use 'letterpaper' for US letter
\documentclass[11pt, a4paper]{awesome-cv}

% Configure page margins with geometry
\geometry{left=1.4cm, top=.8cm, right=1.4cm, bottom=1.8cm, footskip=.5cm}

% Specify the location of the included fonts
\fontdir[fonts/]

% Color for highlights
% Awesome Colors: awesome-emerald, awesome-skyblue, awesome-red, awesome-pink, awesome-orange
%                 awesome-nephritis, awesome-concrete, awesome-darknight
\colorlet{awesome}{awesome-red} 
% Uncomment if you would like to specify your own color
% \definecolor{awesome}{HTML}{CA63A8}

% Colors for text
% Uncomment if you would like to specify your own color
% \definecolor{darktext}{HTML}{414141}
% \definecolor{text}{HTML}{333333}
% \definecolor{graytext}{HTML}{5D5D5D}
% \definecolor{lighttext}{HTML}{999999}

% Set false if you don't want to highlight section with awesome color
\setbool{acvSectionColorHighlight}{true}

% If you would like to change the social information separator from a pipe (|) to something else
\renewcommand{\acvHeaderSocialSep}{\quad\textbar\quad}


%-------------------------------------------------------------------------------
%	PERSONAL INFORMATION
%	Comment any of the lines below if they are not required
%-------------------------------------------------------------------------------
% Available options: circle|rectangle,edge/noedge,left/right
% \photo{profile.png}
\name{Nestor D.}{Pereira Neto}
\position{Electronic engineer{\enskip\cdotp\enskip}36}
\address{Street Eduardo Campos n$^{\circ}$10, Boca do Rio, Salvador - BA, Brazil, 41705-230}

\mobile{(71) 9 9698-6755}
\email{nestor-dp@hotmail.com}
%\homepage{www.posquit0.com}
\github{NestorDP}
\linkedin{nestorpneto}
%\gplus{+NestorNeto85}
\lattes{lattes.cnpq.br/8490310764316084}

% \gitlab{gitlab-id}
% \stackoverflow{SO-id}{SO-name}
% \twitter{@twit}
% \skype{skype-id}
% \reddit{reddit-id}
% \extrainfo{extra informations}

%\quote{``Be the change that you want to see in the world."}

\quote{``I have a degree in Electronic Engineering (2013) and a specialization in Biomedical Engineering (2019). I have experience in the maintenance of diagnostic imaging hospital equipment. In 2011, I started to work as a teacher in technical and higher education courses. I am Master in Electrical Engineering in the computing and robotics area, at the Federal University of Bahia. Since 2019 I have had the pleasure of working in research and development in the field of robotics. Main areas of interest: Hardware decription, Hardware an software integration, C/C++ Programming; Embedded systems; Real-time systems and Mobile robotics localization and navigation.''}
%-------------------------------------------------------------------------------
\begin{document}

% Print the header with above personal informations
% Give optional argument to change alignment(C: center, L: left, R: right)
\makecvheader

% Print the footer with 3 arguments(<left>, <center>, <right>)
% Leave any of these blank if they are not needed
\makecvfooter
  {\today}
  {Nestor D. Pereira Neto~~~·~~~Curriculum Vitae}
  {\thepage}


%-------------------------------------------------------------------------------
%	CV/RESUME CONTENT
%	Each section is imported separately, open each file in turn to modify content
%-------------------------------------------------------------------------------
%-------------------------------------------------------------------------------
%	SECTION TITLE
%-------------------------------------------------------------------------------
\cvsection{Education}


%-------------------------------------------------------------------------------
%	CONTENT
%-------------------------------------------------------------------------------
\begin{cventries}

%---------------------------------------------------------
  \cventry
    {Doctorate Degree in Electrical Engineering - Research Line: Computing and Robotics} % 
    {Federal University of Bahia - UFBA} % Institution
    {Salvador, BA} % Location
    {Fev. 2023 - Mar. 2027} % Date(s)
    {
      \begin{cvitems} % Description(s) bullet points
        \item {Development triggered event algorithm based on FPGA for the ATLAS experiment at the Large Hadron Collider (LHC-CERN).}
      \end{cvitems}
    }


%---------------------------------------------------------
  \cventry
    {Master's Degree in Electrical Engineering - Research Line: Computing and Robotics} % 
    {} % Institution
    {} % Location
    {Apr. 2018 - Dec. 2022} % Date(s)
    {
      \begin{cvitems} % Description(s) bullet points
        \item {Communication between \textit{Robot Operating System - ROS} and \textit{System-on-a-chip - SoC} with integrated FPGA.}
        % \item {Scholarship from Higher Education Personal Improvement Coordination - CAPES }
      \end{cvitems}
    }


%---------------------------------------------------------
  \cventry
    {Specialization in Biomedical Engineering} % 
    {Estácio de Sá University} % Institution
    {Salvador, BA} % Location
    {Oct. 2017 - Feb. 2019} % Date(s)
    {
      \begin{cvitems} % Description(s) bullet points
        \item {Convolutional Neural Network for Detection of QRS Complexes in Electrocardiogram Signals}
      \end{cvitems}
    }


%---------------------------------------------------------
  \cventry
    {Bachelor's degree in Electronic Engineering} % Degree
    {Faculty of Science and Technology - ÁREA1} % Institution
    {Salvador, BA} % Location
    {Feb. 2008 - Jul. 2013} % Date(s)
    {
      \begin{cvitems} % Description(s) bullet points
        \item {Command prototype to fit X-ray equipment to the standards required by the National Health Surveillance Agency - ANVISA.}
        \item {Undergraduate research scholarship for two years}
      \end{cvitems}
    }

\end{cventries}

%-------------------------------------------------------------------------------
%	SECTION TITLE
%-------------------------------------------------------------------------------
\cvsection{Professional Experiences}


%-------------------------------------------------------------------------------
%	CONTENT
%-------------------------------------------------------------------------------
\begin{cventries}

%---------------------------------------------------------
\cventry
  {Specialist II - Robotic Engineering} % Job title
  {SENAI - CIMATEC} % Organization
  {Salvador, BA} % Location
  {Apr. 2022 - currently} % Date(s)
  {
    \begin{cvitems} % Description(s) of tasks/responsibilities
      \item {Development robotics solutions using the ROS2 with expertise in Gazebo Simulation, ROS2 control, SLAM and URDF/Xacro.}
      \item {Software and firmware development using C/C++ and python, to ensure integration and functionality of robotic systems.}
      \item {Collaborating closely with cross-functional teams guiding trainees and fellows.}
    \end{cvitems}
  }

%---------------------------------------------------------
\cventry
  {Consultant II - Robotic Engineering} % Job title
  {} % Organization
  {} % Location
  {Apr. 2019 - Apr. 2022} % Date(s)
  {
    \begin{cvitems} % Description(s) of tasks/responsibilities
      \item {Development robotics solutions using the ROS2 with expertise in Gazebo Simulation, ROS2 control, SLAM and URDF/Xacro.}
      \item {Software and firmware development using C/C++ and python, to ensure integration and functionality of robotic systems.}
    \end{cvitems}
  }

%---------------------------------------------------------
  \cventry
    {Professor part time - 20h} % Job title
    {} % Organization
    {} % Location
    {Nov. 2018 - Apr. 2019} % Date(s)
    {
      \begin{cvitems} % Description(s) of tasks/responsibilities
        \item {Teach class for the course: Industrial Mechatronics. Subjects: Analog Electronics and Digital Eletronic.}
      \end{cvitems}
    }

%---------------------------------------------------------
%  \cventry
%    {Professor horista} % Job title
%    {SENAI - CETIND} % Organization
%    {Lauro de Freitas, BA} % Location
%    {Set. 2017 - Dez. 2017} % Date(s)
%    {
%      \begin{cvitems} % Description(s) of tasks/responsibilities
%        \item {Ministrar aulas para o cursos: Técnico em Mecânica (CHP) e Técnico em Eletroeletrônica (CHP).}
%        \item {Disciplinas: Eletrônica Básica - 32h e Eletrotécnica - 88h.}
%      \end{cvitems}
%    }
    
%---------------------------------------------------------
  \cventry
    {Laboratory technician} % Job title
    {ÁREA1|Wyden College} % Organization
    {Salvador, BA} % Location
    {Jul. 2013 - Aug. 2017} % Date(s)
    {
      \begin{cvitems} % Description(s) of tasks/responsibilities
        \item {Prepare and teach practical classes for the courses: Computer Engineering, Electrical Engineering and Automation Engineering.}
        \item {Test new tools used in laboratories and provide training.}
        \item {Provide extension courses and the experience program.}
      \end{cvitems}
    }

%---------------------------------------------------------
  \cventry
    {Engineer - Technical responsible} % Job title
    {Alfamed Eletromedicina} % Organization
    {Salvador, BA} % Location
    {Aug. 2013 - Feb. 2015} % Date(s)
    {
      \begin{cvitems} % Description(s) of tasks/responsibilities
        \item {Technical responsible for the company.}
        \item {Electronic maintenance in hospital bio-imaging equipment.}
      \end{cvitems}
    }
    
%---------------------------------------------------------
%  \cventry
%    {Professor Temporário} % Job title
%    {CPC Treinamentos} % Organization
%    {Camaçari, BA} % Location
%    {Jan. 2013 - jun. 2013} % Date(s)
%    {
%      \begin{cvitems} % Description(s) of tasks/responsibilities
%        \item {Ministrar aulas das disciplinas: Eletricidade I, Eletrônica Analógica  e Eletrônica Digital.}
%      \end{cvitems}
%    }

%---------------------------------------------------------
  \cventry
    {Teacher - 40h} % Job title
    {Centro Territorial de Educação Profissional da Região Metropolitana CETEP-RM} % Organization
    {Camaçari, BA} % Location
    {jun. 2011 - jun. 2013} % Date(s)
    {
      \begin{cvitems} % Description(s) of tasks/responsibilities
        \item {Teach classes in the following subjects: Analog and digital electronics; Microcontrollers; Electric circuits.}
        \item {Guide students in their course completion work.}
      \end{cvitems}
    }
    
%---------------------------------------------------------
%  \cventry
%    {Analista Técnico} % Job title
%    {TELTRONIC-Brasil} % Organization
%    {Salvador, BA} % Location
%    {Fev. 2011 - Jun. 2011} % Date(s)
%    {
%      \begin{cvitems} % Description(s) of tasks/responsibilities
%        \item {Manutenção corretiva em terminais digitais (móveis e fixos) de comunicação profissional.}
%      \end{cvitems}
%    }
    
%---------------------------------------------------------
  \cventry
    {Maintenance technician} % Job title
    {Micro Comércio e Serviço Ltda.} % Organization
    {Salvador, BA} % Location
    {Jun. 2009 - Sep. 2010} % Date(s)
    {
      \begin{cvitems} % Description(s) of tasks/responsibilities
        \item {Preventive and corrective maintenance in hospital bioimaging equipment.}
      \end{cvitems}
    }

%---------------------------------------------------------
  \cventry
    {Intern - maintenance} % Job title
    {} % Organization
    {} % Location
    {Mar. 2008 - Jun. 2009} % Date(s)
    {
      \begin{cvitems} % Description(s) of tasks/responsibilities
        \item {Preventive and corrective maintenance in hospital bioimaging equipment.}
      \end{cvitems}
    }


%---------------------------------------------------------
\end{cventries}

%-------------------------------------------------------------------------------
%	SECTION TITLE
%-------------------------------------------------------------------------------
\cvsection{Languages}


%-------------------------------------------------------------------------------
%	CONTENT
%-------------------------------------------------------------------------------
\begin{cvskills}

%---------------------------------------------------------
  \cvskill
    {Portuguese} % Category
    {Native.} % Skills

%---------------------------------------------------------
  \cvskill
    {English} % Category
    {B1 level - CERF.} % Skills
    
%---------------------------------------------------------
%  \cvskill
%    {Espanhol} % Category
%    {Nível intermediário.} % Skills


%---------------------------------------------------------
\end{cvskills}


%---------------------------------------------------------
%  \cvskill
%    {Programming} % Category
%    {Python, Node.JS, C/C++, Scala, JAVA, OCaml, LaTeX} % Skills

%---------------------------------------------------------
%  \cvskill
%    {Web} % Category
%    {Django, Express, Redux, React, HTML5, LESS, SASS} % Skills

%---------------------------------------------------------
%  \cvskill
%    {Languages} % Category
%    {Korean, English, Japanese, Chinese} % Skills

%-------------------------------------------------------------------------------
%	SECTION TITLE
%-------------------------------------------------------------------------------
\cvsection{Cursos}


%-------------------------------------------------------------------------------
%	CONTENT
%-------------------------------------------------------------------------------
\begin{cvhonors}

%---------------------------------------------------------
\cvhonor
  {Treinamento em Tecnologia FPGA INTEL (20h)} % Position
  {Macnica - Centro Oficial de Treinamento FPGA INTEL/Altera} % Committee
  {Florianópolis, SC} % Location
  {2021} % Date(s)

%---------------------------------------------------------
  \cvhonor
    {NucLi - Inglês. Compreensão horal (32h)} % Position
    {Universidade Federal da Bahia - UFBA} % Committee
    {Salvador, BA} % Location
    {2018} % Date(s)
    
%---------------------------------------------------------
%  \cvhonor
%   {My English Online - Nível 4 (120h)} % Position
%    {Programa Idiomas sem Fronteiras - MEC.} % Committee
%    {Brasil} % Location
%    {2018} % Date(s)

%---------------------------------------------------------
  \cvhonor
    {Estudo do Controle de Dispositivos via portas paralela, serial e USB (60h)} % Curso
    {Faculdade ÁREA1.} % Instituição
    {Salvador, BA} % Location
    {2010} % Date(s)

%---------------------------------------------------------
  \cvhonor
    {Eletrônica Analógica Aplicada (30h)} % Curso
    {Faculdade ÁREA1.} % Instituição
    {Salvador, BA} % Location
    {2010} % Date(s)

%---------------------------------------------------------
\end{cvhonors}

%-------------------------------------------------------------------------------
%	SECTION TITLE
%-------------------------------------------------------------------------------
\cvsection{Projects and Research}


%-------------------------------------------------------------------------------
%	CONTENT
%-------------------------------------------------------------------------------
\begin{cventries}

%---------------------------------------------------------
  \cventry
    {SuBot - CTG Brasil (China Three Gorges Corporation)} % Affiliation/role
    {SENAI CIMATEC} % Organization/group
    {Salvador, BA} % Location
    {Out. 2021 - Atualmente} % Date(s)
    {
      \begin{cvitems} % Description(s) of experience/contributions/knowledge
        \item {Mobile robotics development project for inspection of high voltage substations.}
        \item {Development of ROS nodes in C/C++ and Python language.}
        \item {Hardware and firmware development for actuator systems and peripheral communication.}
        \item {Electronic project development.}
      \end{cvitems}
    }


%---------------------------------------------------------
  \cventry
    {DIGISUB - Petrobras} % Affiliation/role
    {} % Organization/group
    {} % Location
    {Abrp. 2019 - Oct. 2021} % Date(s)
    {
      \begin{cvitems} % Description(s) of experience/contributions/knowledge
        \item {3D digitizer development project of underwater surfaces in deep water.}
        \item {Development of ROS nodes in C/C++ and Python language.}
        \item {Hardware and firmware development for actuator systems and peripheral communication.}
        \item {Prototype power system sizing: sources, AC-DC/DC-DC converters, batteries.}
      \end{cvitems}
    }


%---------------------------------------------------------
  \cventry
    {Projeto de Mestrado} % Affiliation/role
    {Escola Politécnica da Universidade Federal da Bahia - UFBA} % Organization/group
    {Salvador, BA} % Location
    {Dez. 2018 - atualmente} % Date(s)
    {
      \begin{cvitems} % Description(s) of experience/contributions/knowledge
        \item {1 GigE communication development between FPGA and ROS.}
        \item {Hardware project in verilog for implementation in SOC/FPGA (Cyclone V - Intel).}
        \item {Socket programming in C/C++ language for linux.}
        \item {Development of ROS nodes in C/C++ language.}
        \item {Paper presented at \underline{\href{https://ieee-lascas.org/}{\emph{IBERCHIP 2022}}}.}
      \end{cvitems}
    }

%---------------------------------------------------------
  \cventry
    {Iniciação Científica} % Affiliation/role
    {Faculdade ÁREA1} % Organization/group
    {Salvador, BA} % Location
    {Jul. 2011 - Jul. 2013} % Date(s)
    {
      \begin{cvitems} % Description(s) of experience/contributions/knowledge
        \item {Development of a microcontrolled command to fit old X-ray equipment to current technical standards.}
        \item {Programming 8051 family microcontrollers in assembly language.}
        \item {Research of the technical standards established by ANVISA in relation to X-ray equipment.}
        \item {Presentation of monthly reports with activities developed in the period.}
        \item {Publication of the results in an article in the journal \underline{\href{https://cientefico.emnuvens.com.br/cientefico/article/view/58}{\emph{Cientefico ISSN 1677-1591, jul/dez 2013}}}.}
      \end{cvitems}
    }
    

%---------------------------------------------------------
\end{cventries}


%-------------------------------------------------------------------------------
%	SECTION TITLE
%-------------------------------------------------------------------------------
\cvsection{Competências e Habilidades}


%-------------------------------------------------------------------------------
%	CONTENT
%-------------------------------------------------------------------------------
\begin{cvskills}

%---------------------------------------------------------
  \cvskill
    {Básico} % Category
    {Verilog HDL, Cyclone V, Sockets, Cmake, GDB, Nios II, Embedded Linux, SimuLink.} % Skills

%---------------------------------------------------------
  \cvskill
    {Intermediário} % Category
    {Python, 3D CAD, Quartus Prime, RTOS, OOP, Linux, Git/Github, PIC, AutoCAD, ROS/Gazebo.} % Skills

%---------------------------------------------------------
  \cvskill
    {Avançado} % Category
    {PCB/KiCad, Matlab, C/C++, AVR, 8051, ARM Cortex M, GNU-Make, LaTeX.}
     % Skills

%---------------------------------------------------------
\end{cvskills}


%---------------------------------------------------------
%  \cvskill
%    {Programming} % Category
%    {Python, Node.JS, C/C++, Scala, JAVA, OCaml, LaTeX} % Skills

%---------------------------------------------------------
%  \cvskill
%    {Web} % Category
%    {Django, Express, Redux, React, HTML5, LESS, SASS} % Skills

%---------------------------------------------------------
%  \cvskill
%    {Languages} % Category
%    {Korean, English, Japanese, Chinese} % Skills
%%-------------------------------------------------------------------------------
%	SECTION TITLE
%-------------------------------------------------------------------------------
\cvsection{Honors \& Awards}


%-------------------------------------------------------------------------------
%	SUBSECTION TITLE
%-------------------------------------------------------------------------------
% \cvsubsection{International}


%-------------------------------------------------------------------------------
%	CONTENT
%-------------------------------------------------------------------------------
\begin{cvhonors}

%---------------------------------------------------------
  \cvhonor
    {PETROBRAS} % Award
    {Prêmio Inventor 2022 - Award for filing a patent for an underwater 3D scanner} % Event
    {Rio de Janeiro, Brasil} % Location
    {2022} % Date(s)


%---------------------------------------------------------
\end{cvhonors}


%-------------------------------------------------------------------------------
%	SUBSECTION TITLE
%-------------------------------------------------------------------------------
% \cvsubsection{Domestic}


%-------------------------------------------------------------------------------
%	CONTENT
%-------------------------------------------------------------------------------
% \begin{cvhonors}

%---------------------------------------------------------
%   \cvhonor
%     {3rd Place} % Award
%     {WITHCON Hacking Competition Final} % Event
%     {Seoul, S.Korea} % Location
%     {2015} % Date(s)


% \end{cvhonors}

%-------------------------------------------------------------------------------
%	SECTION TITLE
%-------------------------------------------------------------------------------
\cvsection{Presentation}


%-------------------------------------------------------------------------------
%-------------------------------------------------------------------------------
\begin{cventries}

%---------------------------------------------------------
  \cventry
    {IEEE Circuits and Systems Society in Latin Ameriaca} % Role
    {28th IBERCHIP Workshop} % Event
    {Santiago, Chile} % Location
    {1-4 Mar. 2022} % Date(s)
    {
      \begin{cvitems} % Description(s)
        \item {Comunicação entre Robot Operating System - ROS e SoC com FPGA integrado}
      \end{cvitems}
    }

%---------------------------------------------------------
  % \cventry
  %   {Presenter for <Metasploit 101>} % Role
  %   {6th Hacking Camp - S.Korea} % Event
  %   {S.Korea} % Location
  %   {Sep. 2012} % Date(s)
  %   {
  %     \begin{cvitems} % Description(s)
  %       \item {Introduced basic procedure for penetration testing and how to use Metasploit}
  %     \end{cvitems}
  %   }

%---------------------------------------------------------
\end{cventries}

%\input{cv/writing.tex}



%-------------------------------------------------------------------------------
\end{document}
